\section{Einleitung}
Moderne Softwareanwendungen müssen zuverlässig, wartbar und fehlertolerant sein, da sie häufig die Grundlage komplexer Systeme, beispielsweise in Unternehmen, bilden. Trotz sorgfältiger Planung lassen sich Laufzeitfehler wie ungültige Benutzereingaben oder fehlende Dateien nicht vollständig vermeiden. Um mit solchen Situationen kontrolliert umzugehen, wird in der Softwareentwicklung das sogenannte Exception Handling, also die Fehlerbehandlung, gezielt eingesetzt. Zu den wichtigsten Mechanismen gehören dabei die Konstrukte try/catch, finally sowie throws und die Möglichkeit zur Definition eigener Fehlerklassen.

Die Motivation für diese Arbeit ergibt sich aus ihrer hohen praktischen Relevanz. In vielen Projekten werden Fehler gar nicht oder nur unzureichend behandelt, was zu unkontrollierten Programmabstürzen führen kann. Obwohl Exception Handling ein fester Bestandteil moderner Programmiersprachen ist, kann die Anwendung davon oft unterschätzt oder falsch umgesetzt werden. Ein fundiertes Verständnis der zugrunde liegenden Konzepte ist daher essenziell, um robuste und stabile Anwendungen zu entwickeln.

Ausgehend von dieser Problematik ergibt sich folgende Forschungsfrage: Wie kann Exception Handling in Java effektiv eingesetzt werden, um eine strukturierte Fehlerbehandlung und die Stabilität von Softwareanwendungen zu gewährleisten? Ziel der Arbeit ist es, grundlegende Konzepte des Exception Handlings zunächst zu erläutern und deren praxisnahe Anwendung anhand eines Beispielprojekts nachvollziehbar darzustellen. Dabei soll demonstriert werden, wie durch den gezielten Einsatz der verfügbaren Mechanismen eine effektive Fehlerbehandlung erreicht werden kann.

Zur Beantwortung der Forschungsfrage wird ein theoretisch-praktischer Ansatz verfolgt. Dazu werden zunächst die Grundlagen des Exception Handlings erläutert. Darauf aufbauend werden typische Problembereiche sowie die Anwendungsrelevanz der Fehlerbehandlung anhand von Beispielen thematisiert. Abschließend erfolgt ein Theorie-Praxis-Transfer, bei dem die entwickelten Konzepte in einem Java-Programm umgesetzt, erläutert und kritisch reflektiert werden.